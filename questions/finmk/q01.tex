% questions/finmk/q01.tex

\SetQuestionHeader{Финансовые рынки и финансово-кредитные институты}{topics:finmk}

\section*{Вопрос 1. Финансовый рынок: структура и его роль в экономике}
\phantomsection
\label{q:finmk:01}

\begin{mytheo}{}{}
    \textbf{Финансовый рынок} – это совокупность экономических отношений по поводу купли-продажи финансовых активов (денежных средств, ценных бумаг, производных финансовых инструментов и т.п.), а также институтов и инфраструктуры, обеспечивающих эти операции. В структуре финансовой системы он выступает ключевым звеном, через которое происходит перераспределение финансовых ресурсов между секторами экономики.
\end{mytheo}

С точки зрения структуры можно выделить следующие уровни:
\begin{itemize}
    \item денежный рынок – операции с высоколиквидными краткосрочными инструментами (до одного года): межбанковские кредиты, краткосрочные облигации, векселя, депозиты и т.п.;
	\item рынок капитала – операции с долгосрочными инструментами (свыше года): акции, долгосрочные облигации, долевые участия и другие схожие инструменты;
    \item валютный рынок – операции с иностранной валютой и валютными деривативами;
    \item рынок производных финансовых инструментов – операции с деривативами (фьючерсы, опционы, свопы и т.п.), которые базируются на других финансовных активах.
\end{itemize}