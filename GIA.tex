%!TEX program = lualatex
\documentclass[12pt]{extarticle}

% --------- ПОЛЯ ---------
\usepackage[a4paper, left=2cm, right=2cm, top=2cm, bottom=2cm]{geometry}

% --------- ЯЗЫКИ ---------
\usepackage{polyglossia}
\setdefaultlanguage{russian}
\setotherlanguage{english}

% --------- ШРИФТЫ ---------
\usepackage{fontspec}

\setmainfont{CMU Serif}
\setsansfont{CMU Sans Serif}
\setmonofont{CMU Typewriter Text}

\newfontfamily\OpenSans{Open Sans}[Ligatures=TeX, Scale=1.05]

% --------- ИНТЕРВАЛЫ / ВЫРАВНИВАНИЕ ---------
\usepackage{setspace}
\onehalfspacing

\usepackage{ragged2e}
\justifying

% --------- ПЕРЕНОСЫ ВЫКЛ ---------
\usepackage[none]{hyphenat}
\hyphenpenalty=10000
\exhyphenpenalty=10000
\emergencystretch=1em

% --------- ЦВЕТА / ССЫЛКИ ---------
\usepackage{xcolor}
\definecolor{handbookblue}{HTML}{003366}
\definecolor{linkorange}{HTML}{E67E22}

\usepackage[hidelinks]{hyperref}

% --------- ОГЛАВЛЕНИЕ ---------
\usepackage{tocloft}
\renewcommand{\cfttoctitlefont}{\sffamily\bfseries\Huge}
\renewcommand{\cftaftertoctitle}{\par\vskip1em}
\renewcommand{\cftsecfont}{\sffamily\bfseries}
\renewcommand{\cftsecpagefont}{\sffamily\bfseries}
\setlength{\cftbeforesecskip}{0.5em}
\setlength{\cftsecindent}{0pt}
\setlength{\cftsecnumwidth}{0pt}
\renewcommand{\cftsecleader}{\hfill}

% --------- ЗАГОЛОВКИ КАК «ССЫЛКИ» ПО ЦЕНТРУ ---------
\usepackage{titlesec}
\titleformat{\section}
  {\OpenSans\normalsize\bfseries\centering}
  {}
  {0pt}
  {}
\titlespacing*{\section}{0pt}{1.5ex plus 0.5ex}{1ex plus 0.2ex}

% --------- КОЛОНТИТУЛЫ / НАВИГАЦИЯ ---------
\usepackage{fancyhdr}
\pagestyle{plain}

% базовый стиль
\newcommand{\SetDefaultHeader}{%
  \fancyhf{}%
  \fancyfoot[C]{\thepage}%
  \renewcommand{\headrulewidth}{0.4pt}%
}

% ЛЕВАЯ ЧАСТЬ: название раздела (или вопроса),
% ПРАВАЯ ЧАСТЬ: две кнопки – Назад / К содержанию
% #1 — текст слева (название блока или вопрос),
% #2 — label страницы с ТЕМАМИ раздела (для "Назад к разделу")
\newcommand{\SetQuestionHeader}[2]{%
  \fancyhf{}%
  \fancyfoot[C]{\thepage}%
  \renewcommand{\headrulewidth}{0.4pt}%
  \fancyhead[C]{%
    \makebox[\textwidth]{%
      \parbox[t]{0.5\textwidth}{%
        \raggedright\small #1%
      }%
      \parbox[t]{0.5\textwidth}{%
        \raggedleft\small
        \textcolor{linkorange}{\hyperref[#2]{Назад к разделу}}\\[-0.2em]
        \textcolor{linkorange}{\hyperref[toc-anchor]{К содержанию}}%
      }%
    }%
  }%
}

% ---------------- ДОКУМЕНТ ----------------
\begin{document}

% --------- ТИТУЛЬНИК ---------
\begin{titlepage}
    \pagecolor{handbookblue}
    \color{white}
    \centering

    {\large
    РУДН\\
    Экономический факультет\\
    38.04.08 «Финансы и кредит»\\
    «Современные финансовые технологии в инвестировании\\
    и банковском бизнесе»\\[10em]
    }

    {\Huge\textbf{ХЭНДБУК}\\[0.5em]
     {\Large\textbf{ДЛЯ ГИА}}\\[4em]
    }

    {\normalsize Москва, 2025}

\end{titlepage}

\nopagecolor
\color{black}

% --------- ОГЛАВЛЕНИЕ (якорь для "К содержанию") ---------
\clearpage
\pagestyle{fancy}
\SetDefaultHeader

\phantomsection
\label{toc-anchor}
\tableofcontents
\clearpage

% --------- ТЕОРЕТИЧЕСКАЯ ЧАСТЬ: КЛИКАБЕЛЬНЫЙ ЛИСТ ---------
\SetDefaultHeader
\section*{ТЕОРМИН \\ Теоретическая часть}
\phantomsection
\addcontentsline{toc}{section}{ТЕОРМИН Теоретическая часть}
\label{sec:theory}

\textbf{Основная часть государственного экзамена:}
\begin{enumerate}
  \item \textcolor{linkorange}{\hyperref[topics:finmk]{Финансовые рынки и финансово-кредитные институты}}
  \item \textcolor{linkorange}{\hyperref[topics:std]{Международные стандарты управления банковской деятельностью}}
  \item \textcolor{linkorange}{\hyperref[topics:strat]{Стратегии и современная модель управления в сфере денежно-кредитных отношений}}
  \item \textcolor{linkorange}{\hyperref[topics:port]{Управление инвестиционным портфелем}}
  \item \textcolor{linkorange}{\hyperref[topics:fe]{Финансовый инжиниринг}}
  \item \textcolor{linkorange}{\hyperref[topics:val]{Оценка и управление стоимостью банка}}
  \item \textcolor{linkorange}{\hyperref[topics:inv]{Инвестиционный менеджмент}}
\end{enumerate}

\clearpage

% ======================================================
% 1. ФИНАНСОВЫЕ РЫНКИ И ФК ИНСТИТУТЫ
% ======================================================

% --- Лист: предмет + темы ---
\SetDefaultHeader
\section*{Финансовые рынки и финансово-кредитные институты}
\phantomsection
\addcontentsline{toc}{section}{Финансовые рынки и финансово-кредитные институты}
\label{topics:finmk}

\textbf{Темы (вопросы):}
\begin{enumerate}
  \item \hyperref[q:finmk:01]{Финансовый рынок: структура и его роль в экономике.}
  \item \hyperref[q:finmk:02]{Международные финансовые рынки. Особенности развития в современных условиях.}
  \item \hyperref[q:finmk:03]{Инфраструктура финансового рынка и её основные составляющие.}
  \item \hyperref[q:finmk:04]{Классификация финансово-кредитных институтов.}
  \item \hyperref[q:finmk:05]{Особенности деятельности финансово-кредитных институтов на финансовых рынках.}
  \item \hyperref[q:finmk:06]{Банки как основные участники рынка кредитных ресурсов.}
  \item \hyperref[q:finmk:07]{Институциональные инвесторы: паевые и акционерные фонды, негосударственные пенсионные фонды.}
  \item \hyperref[q:finmk:08]{Регулирование профессиональной деятельности участников финансового рынка.}
  \item \hyperref[q:finmk:09]{Роль Центрального банка в регулировании финансового рынка.}
  \item \hyperref[q:finmk:10]{Рынок золота как особый сегмент финансового рынка.}
  \item \hyperref[q:finmk:11]{Механизм функционирования рынка золота.}
  \item \hyperref[q:finmk:12]{Валютный рынок и валютные операции. Конвертируемость валют.}
  \item \hyperref[q:finmk:13]{Влияние динамики валютного курса на экономику.}
  \item \hyperref[q:finmk:14]{Мировые финансовые центры: классификация и их характеристика.}
\end{enumerate}

\clearpage

% --- Пример листа с вопросами для этого блока ---
\SetQuestionHeader{Финансовые рынки и финансово-кредитные институты}{topics:finmk}
\section*{Финансовые рынки и финансово-кредитные институты: вопросы}
\begin{enumerate}
  \item \label{q:finmk:01} Финансовый рынок: структура и его роль в экономике.
  \item \label{q:finmk:02} Международные финансовые рынки. Особенности развития в современных условиях.
  \item \label{q:finmk:03} Инфраструктура финансового рынка и её основные составляющие.
  \item \label{q:finmk:04} Классификация финансово-кредитных институтов.
  \item \label{q:finmk:05} Особенности деятельности финансово-кредитных институтов на финансовых рынках.
  \item \label{q:finmk:06} Банки как основные участники рынка кредитных ресурсов.
  \item \label{q:finmk:07} Институциональные инвесторы: паевые и акционерные фонды, негосударственные пенсионные фонды.
  \item \label{q:finmk:08} Регулирование профессиональной деятельности участников финансового рынка.
  \item \label{q:finmk:09} Роль Центрального банка в регулировании финансового рынка.
  \item \label{q:finmk:10} Рынок золота как особый сегмент финансового рынка.
  \item \label{q:finmk:11} Механизм функционирования рынка золота.
  \item \label{q:finmk:12} Валютный рынок и валютные операции. Конвертируемость валют.
  \item \label{q:finmk:13} Влияние динамики валютного курса на экономику.
  \item \label{q:finmk:14} Мировые финансовые центры: классификация и их характеристика.
\end{enumerate}

\clearpage

% ======================================================
% 2. МЕЖДУНАРОДНЫЕ СТАНДАРТЫ УПРАВЛЕНИЯ БАНКОВСКОЙ ДЕЯТЕЛЬНОСТЬЮ
% ======================================================

% --- Лист: предмет + темы ---
\SetDefaultHeader
\section*{Международные стандарты управления банковской деятельностью}
\phantomsection
\addcontentsline{toc}{section}{Международные стандарты управления банковской деятельностью}
\label{topics:std}

\textbf{Темы (вопросы):}
\begin{enumerate}
  \item Микропруденциальное и макропруденциальное регулирование банковской деятельности: цели, задачи, принципы.
  \item Роль БКБН в продвижении новых стандартов регулирования.
  \item Стандарт корпоративного управления в банках: рекомендации БКБН и российская практика.
  \item Стандарт управления банковским капиталом: основные нормативы и показатели. Рекомендации БКБН и российская практика.
  \item Управление кредитным риском на основе подхода оценки внутренних рейтингов (IRB Approach).
  \item Управление риском ликвидности: основные нормативы и показатели.
  \item Рекомендации БКБН и российская практика управления ликвидностью.
  \item Управление рыночным риском: основные нормативы и показатели.
  \item Рекомендации БКБН и российская практика управления рыночным риском.
  \item Подходы к измерению операционного риска: рекомендации БКБН и российская практика.
\end{enumerate}

\clearpage

% (Здесь по аналогии можно сделать страницу "вопросы" с SetQuestionHeader{...}{topics:std})

% ======================================================
% 3. СТРАТЕГИИ И СОВРЕМЕННАЯ МОДЕЛЬ УПРАВЛЕНИЯ В СФЕРЕ ДЕНЕЖНО-КРЕДИТНЫХ ОТНОШЕНИЙ
% ======================================================

\SetDefaultHeader
\section*{Стратегии и современная модель управления в сфере денежно-кредитных отношений}
\phantomsection
\addcontentsline{toc}{section}{Стратегии и современная модель управления в сфере денежно-кредитных отношений}
\label{topics:strat}

\textbf{Темы (вопросы):}
\begin{enumerate}
  \item Роль центрального банка в регулировании денежно-кредитных отношений и развитии банковского сектора.
  \item Особенности стратегии банков с государственным участием в период кризиса и посткризисного развития.
  \item Особенности стратегии кредитования банками с иностранным участием.
  \item Стратегия развития региональной банковской системы и её роль в развитии региональной экономики.
  \item Приоритеты в развитии кредитования экономических субъектов.
  \item Направления совершенствования управления кредитом.
  \item Понятие, содержание и признаки кредитной экспансии.
  \item Классификация кредитной экспансии.
  \item Содержание модели управления кредитом в экономике.
  \item Стратегия развития региональных банков.
  \item Особенности монетарной политики центрального банка в современных условиях.
  \item Стратегии управления денежно-кредитными отношениями в современных условиях.
\end{enumerate}


\clearpage

\end{document}